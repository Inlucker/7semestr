\section*{Условия и исходные данные}

Определить УЗД (уровни звукового давления) в расчетной точке при заданных уровнях звуковой мощности источников (источники ненаправленные), указанном расположении расчетной точки относительно источников шума, габаритных размерах промышленного помещения. Максимальный габарит любого источника много меньше расстояния до расчетной точки. Полученные данные сравнить с нормативными значениями (СП 51.13330.2011). Построить расчетный и предельный спектры. Сделать выводы о необходимости защитных мероприятий. Предложить защитные мероприятия.

Примечание: постоянную помещения В определить в соответствии с назначением помещения и его объемом по СНиП II-12-77

\begin{table}[H]
	\caption{Уровни звуковой мощности источников}
	\label{tab:tab1}
	\raggedleft\tiny
	\resizebox{\columnwidth}{!}{
		\renewcommand{\arraystretch}{1.4}
		\begin{tabular}{|c|c|c|c|c|c|c|c|c|}
			\hline
			\multirow{2}{*}{№} & \multicolumn{8}{c|}{$L_{p} = f(f_{\text{сг}})$, дБ} \\
			\hhline{~--------}
			& 63 & 125 & 250 & 500 & 1000 &  2000 & 4000 & 8000 \\
			\hline
			1 & 80 & 84 & 83 & 87 & 84 & 82 & 94 & 96 \\
			\hline
			2 & 83 & 87 & 85 & 85 & 85 & 82 & 83 & 83 \\
			\hline
			3 & 78 & 81 & 83 & 85 & 85 & 86 & 89 & 85 \\
			\hline
	\end{tabular}}
\end{table}

\begin{table}[H]
	\caption{Предельно допустимые УЗД в каждой октавной полосе}
	\label{tab:tab2}
	\raggedleft\tiny
	\resizebox{\columnwidth}{!}{
		\renewcommand{\arraystretch}{1.4}
		\begin{tabular}{|c|c|c|c|c|c|c|c|c|}
			\hline
			\multirow{2}{*}{№} & \multicolumn{8}{c|}{Предельно допустимые УЗД в каждой октавной полосе, дБ} \\
			\hhline{~--------}
			& 63 & 125 & 250 & 500 & 1000 &  2000 & 4000 & 8000 \\
			\hline
			4 & 90 & 82 & 77 & 73 & 70 & 68 & 66 & 64 \\
			\hline
	\end{tabular}}
\end{table}

\section*{Расчет УЗД}

1. Для определения уровня звукового давления (УЗД) в расчетной точке, необходимо найти УЗД в этой точке от каждого источника в каждой октавной полосе по формуле (\ref{eq:eq1}):

\begin{equation}
	\label{eq:eq1}
	L_{i} = L_{wi} + 10 \lg(\frac{\Phi}{S_{i}} + \frac{4}{B})
\end{equation}

\noindent где\\
$\L_{wi}$ --- уровень звуковой мощности i-го источника;\\
$\Phi$ --- фактор направленности источника;\\
$S_{i}$ --- плошадь поверхности излучения i-го источника;\\
$\text{B}$ --- постоянная помещения.

Для определения суммарного УЗД в расчетной точке от $N$ источников используется формула (\ref{eq:eq2}):

\begin{equation}
	\label{eq:eq2}
	L_{\sum} = 10\lg(\sum_{i=1}^{N} 10^{0.1 L_{i}})
\end{equation}

2. Для нахождения плошади поверхности i-го источника используется формула (\ref{eq:eq3}):

\begin{equation}
	\label{eq:eq3}
	S_{i} = \Omega_{i} * R_{i}^{2}
\end{equation}

% TODO
\noindent где  \\
$\Omega_{i}$ --- телесный угол i-го источника; \\
$R_{i}$ --- расстояние между i-м источником и расчетной точкой.

\begin{align*}
	S_{1} = 4\pi * 2^2 \approx 50.27 \\
	S_{2} = \pi * 7^2 \approx 153.94\\
	S_{3} = \frac{\pi}{2}*10^2 \approx 157.08 \\ 
\end{align*}

3. Для нахождения постоянной помещения используется формула (\ref{eq:eq4}):

\begin{equation}
	\label{eq:eq4}
	B = B_{1000} * \mu
\end{equation}

\noindent где \\
$B_{1000}$ --- постоянная помещения в полосе 1000 Гц \\
$\mu$ --- частотный множитель.

Значения постоянной помещения в полосе 1000 Гц и частотного множителя приведены из СНиП II-12-77.

Объем помещения: $V = 15*30*4 = 1800 \quad \text{м}^3$. Тип помещения 1 (с небольшим количеством людей), поэтому $B_{1000} = \frac{V}{20}$.

Объем помещения больше 1000 $\text{м}^3$, поэтому следует воспользоваться следующими значениями частотного показателя (таблица \ref{tab:tab3}):

\begin{table}[H]
	\caption{Значения частотного множителя $\mu$}
	\label{tab:tab3}
	\raggedleft\tiny
	\resizebox{\columnwidth}{!}{
		\renewcommand{\arraystretch}{1.4}
		\begin{tabular}{|c|c|c|c|c|c|c|c|}
			\hline
			\multicolumn{8}{|c|}{$L_{p} = f(f_{\text{сг}})$, дБ} \\
			\hline
			63 & 125 & 250 & 500 & 1000 &  2000 & 4000 & 8000 \\
			\hline
			0.5 & 0.5 & 0.55 & 0.7 & 1 & 1.6 & 3 & 6 \\
			\hline
	\end{tabular}}
\end{table}

Подставляя значения в формулу (\ref{eq:eq4}), получаются следующие значения постоянной помещения (таблица \ref{tab:tab4}):

\begin{table}[H]
	\caption{Значения постоянной помещения $B$}
	\label{tab:tab4}
	\raggedleft\tiny
	\resizebox{\columnwidth}{!}{
		\renewcommand{\arraystretch}{1.4}
		\begin{tabular}{|c|c|c|c|c|c|c|c|}
			\hline
			\multicolumn{8}{|c|}{$L_{p} = f(f_{\text{сг}})$, дБ} \\
			\hline
			63 & 125 & 250 & 500 & 1000 &  2000 & 4000 & 8000 \\
			\hline
			45 & 45 & 49,5 & 63 & 90 & 144 & 270 & 540 \\
			\hline
	\end{tabular}}
\end{table}

4. Результаты подставновки найденных значений в формулы (\ref{eq:eq1}, \ref{eq:eq2}) приведены в таблице \ref{tab:tab5}.

\begin{table}[H]
	\caption{УЗД для каждого источника в каждой октавной полосе, дБ}
	\label{tab:tab5}
	\raggedleft\tiny
	\resizebox{\columnwidth}{!}{
		\renewcommand{\arraystretch}{1.75}
		\begin{tabular}{|c|c|c|c|c|c|c|c|c|}
			\hline
			\multirow{2}{*}{№} & \multicolumn{8}{c|}{$L_{p} = f(f_{\text{сг}})$, дБ} \\
			\hhline{~--------}
			 & 63 & 125 & 250 & 500 & 1000 &  2000 & 4000 & 8000 \\
			\hline			
			$L_{w1}$  & 80 & 84 & 83 & 87 & 84 & 82 & 94 & 96 \\
			\hline
			$L_{w2}$ & 83 & 87 & 85 & 85 & 85 & 82 & 83 & 83 \\
			\hline
			$L_{w3}$ & 78 & 81 & 83 & 85 & 85 & 86 & 89 & 85 \\
			\hline
			$\mu$ & 0.5 & 0.5 & 0.55 & 0.7 & 1 & 1.6 & 3 & 6 \\
			\hline
			B & 45 & 45 & 49,5 & 63 & 90 & 144 & 270 & 540 \\
			\hline
			$L_{1}$ & 70,37 & 74,37 & 73,03 & 76,21 & 72,08 & 68,78 & 79,40 & 80,36 \\
			\hline
			$L_{2}$ & 88.42 & 89.42 & 87.02 & 87.01 & 83.55 & 81.68 & 77.31 & 72.99 \\
			\hline
			$L_{3}$ & 72,79 & 76,79 & 74,41 & 73,45 & 72,07 & 67,35 & 66,29 & 64,43 \\
			\hline
			$L_{\sum}$ & 75,55 & 79,40 & 78,14 & 79,35 & 76,84 & 74,25 & 80,34 & 80,64 \\
			\hline
			$L_{\text{норм}}$ & 90 & 82 & 77 & 73 & 70 & 68 & 66 & 64 \\
			\hline
			$\Delta L$ & -14,45 & -2,60 & 1,14 & 6,35 & 6,84 & 6,25 & 14,34 & 16,64 \\
			\hline
	\end{tabular}}
\end{table}

\noindent где $\Delta L$ = $L_{\sum} - L_{\text{норм}}$.

\imgsc{H}{1}{graph}{Спектр УЗД}

\section*{Вывод}

Судя по результатам, необходимо применить защитные меры, например, уплотнение по периметру притворов окон, ворот, дверей; звукоизоляцию мест пересечения ограждающих конструкций инженерными коммуникациями; устройство звукоизолированных кабин наблюдения и дистанционного управления; укрытий; кожухов. Также можно использовать глушители шума, звукопоглощающие облицовки в газовоздушных трактах вентиляционных систем с механическим побуждением и системы кондиционирования воздуха и газодинамические установоки.


