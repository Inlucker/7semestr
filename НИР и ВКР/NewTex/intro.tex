\StructuredChapter{ВВЕДЕНИЕ}

Уровень удобства использования программного интерфейса влияет на качество всего ПО в целом. Признаком недостаточного уровня удобства использования является наличие проблем взаимодействия пользователя с пользовательским интерфейсом. Они могут быть связаны либо со сложностью формулирования плана действий (принятия решений, что делать дальше), либо с непониманием ответа системы (как изменения в интерфейсе связаны с выполненными действиями) \cite{1}.

Проблемы взаимодействия в большинстве случаев можно определить по наличию в данных активности пользователей определенных последовательностей действий (шаблонов). Для их обнаружения применяются различные методы анализа собираемых данных – как требующие ручного анализа (например, тепловые карты \cite{2,3}), так и использующие алгоритмы автоматического анализа \cite{1} на основе шаблонов, выявленных исследователями ранее \cite{4, 5, 6}. Автоматический анализ экономит время и деньги, так как эксперты вместо анализа всех данных фокусируют внимание на отдельных областях пользовательского интерфейса,
где были выявлены соответствующие шаблоны.

% На настоящий момент в открытых научных источниках не удалось найти формализованное представление данных активности пользователей ПО. В статье представлена разработанная авторами математическая модель активности пользователей ПО. Эта модель может найти применение при оценке удобства пользовательских интерфейсов. Целями являются максимальная формализация оценки удобства использования и формирование критериев для повышения эффективности взаимодействия пользователей с пользовательским интерфейсом.

\textbf{Цель работы} – провести обзор существующих методов анализа пользовательской активности, сформулировать критерии для их оценки и провести сравнение рассмотренных методов.

\textbf{Задачи работы:}
\begin{itemize}
	\item[---] рассмотреть существующие решения в области анализа пользовательской активности;
	\item[---] классифицировать методы анализа пользовательской активности;
	\item[---] выбрать для них критерии оценки и сравнить.
\end{itemize}
