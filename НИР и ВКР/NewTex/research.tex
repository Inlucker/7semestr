\chapter{Сравнение}
\label{cha:research}

В таблице \ref{tab} приведено сравнение рассмотренных методов.

\begin{table}[H]
	\begin{center}
		\caption{Сравнение рассмотренных методов}
		\label{tab}
		""\newline
		\begin{tabular}{ | c | c | c | c | }
			\hline
			Метод & \specialcell{Требование к\\ входным данным}  & \specialcell{Учет\\времени \\транз-ий} & \specialcell{Сложность\\алгоритма} \\ \hline
			
			\specialcell{Мат. модель\\пользов.\\актив. ПО} & \specialcell{Множество событий,\\функция классификации\\событий, множество\\сессий, множество\\последовательных\\шаблонов} & Нет & \specialcell{$O(n \cdot m)$, где $n$ --\\кол-во шаблонов,\\$m$ -- кол-во сессий} \\ \hline
			
			\specialcell{Apriori} & \specialcell{Транзакции с набором\\элементов и\\минимальный уровень\\поддержки} & Нет & \specialcell{$O(|D| \cdot |I| \cdot 2^{|I|})$,\\где $|D|$ -- кол-во\\транзакций,\\$|I|$ -- общее число\\предметов} \\ \hline
			
			\specialcell{GSP} & \specialcell{База данных с полями:\\id последовательности,\\id и время транзакции,\\набор элементов\\и минимальный\\уровень поддержки} & Да & \specialcell{$O(|I|^l)$, где $|I|$ --\\общее число\\предметов,\\$l$ -- длина\\наибольшей ЧВП} \\ \hline
			
			\specialcell{GOMS} & \specialcell{Последовательность\\действий} & Нет & \specialcell{$O(n)$, где n --\\число действий\\в послед-ти} \\ \hline
		\end{tabular}
	\end{center}
\end{table}

Вычисление уровней поддержки шаблонов поведения пользователя позволяет ранжировать их по степени приоритета для детального анализа.
Методы поиска ассоциативных правил и последовательных шаблонов позволяют найти новые (неизвестные ранее) шаблоны взаимодействия пользователей с программным обеспечением.
А анализ временных характеристик, позволяет оценить эффективность взаимодействия пользователя с интерфейсом.