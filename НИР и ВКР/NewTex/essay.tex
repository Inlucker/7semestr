\StructuredChapter{РЕФЕРАТ}

Расчетно-пояснительная записка содержит 32 с., 1 табл., 16 источников.

Ключевые слова: анализ пользовательской активности, последовательные шаблоны, математическая модель активности пользователей ПО, ассоциативные правила.

%Объектом научно-исследовательской работы является пользовательская активность.

%\textbf{Цель работы} – провести обзор существующих методов анализа пользовательской активности, сформулировать критерии для их оценки и провести сравнение рассмотренных методов.
%
%\textbf{Задачи работы:}
%\begin{itemize}
%	\item рассмотреть существующие решения в области анализа пользовательской активности;
%	\item классифицировать методы анализа пользовательской активности;
%	\item выбрать для них критерии оценки и сравнить.
%\end{itemize}

В результате выполнения научно-исследовательской работы был представлен обзор разных подходов к анализу пользовательской активности, рассмотрены и классифицированы существующие методы, а также проведено их сравнение.

%\noindent\textbf{Ключевые слова}: Базы Данных, SQL, PostgreSQL, Киберспорт\\
%
%%Объектом разработки является базы данных для хранения конфигураций нейронных сетей.
%
%\textbf{Цель работы} – реализовать базу данных 	для организации турниров по киберспортивной дисциплине “Dota 2” и их управления.
%
%Для реализации данного проекта, необходимо решить ряд задач:
%\begin{itemize}
%	\item формализовать задание, определить необходимый функционал;
%	\item определить роли пользователей;
%	\item проанализировать существующие аналоги;
%	\item проанализировать модели базы данных;
%	%\item описать структуру базы данных;
%	\item построить инфологическую модель базы данных;
%	\item спроектировать приложение для доступа к базе данных;
%	\item создать и заполнить базу данных;
%	\item реализовать интерфейс для доступа к базе данных;
%	\item разработать программу, реализующую поставленную задачу;
%	% Добавить цель для исследовния
%	\item провести сравнительный анализ времени выполнения запросов к базе данных с использованием индексов и без.
%\end{itemize}

%В результате выполнения работы была спроектирована и разработана база данных для хранения конфигураций нейронных сетей.
%
%По результатам экспериментальных измерений, использование кэширования при получении информации из базы данных позволяет снизить времени отклика системы вплоть до 39 раз, при условии, что запрашиваемая информация находится в кэше.