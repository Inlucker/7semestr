\chapter{Аналитическая часть}
\label{cha:analysis}

Тестирование удобства использования программного обеспечения обычно состоит из двух этапов. Первый этап заключается в сборе данных о действиях, совершаемых пользователями посредством взаимодействия с графическим интерфейсом программы (движение курсора мыши, нажатие клавиш мыши, нажатие клавиш клавиатуры и т.д.), и характеристиках действий (координаты курсора, частота нажатия, используемые клавиши и т.д.). Такие данные обозначаются устоявшимся термином «активность пользователей». Второй этап – анализ этих данных экспертом с целью выявления проблем связанных с удобством использования, что является трудоемкой задачей. Поэтому, встает вопрос о хотя бы частичной автоматизации этого этапа, для чего требуется наличие соответствующих моделей и алгоритмов.

\section{Шаблоны поведения пользователя}
По мнению многих исследователей (например, авторов \cite{1,4,5,6}), индикатором проблем удобства использования может являться наличие часто повторяемых одинаковых последовательностей действий. Они могут означать, что пользователь пытается достичь цели и каждый раз терпит неудачу. Например, пользователь пытается взаимодействовать с изображением, которое он принял за кнопку \cite{1}, или пользователь нажимает кнопку и каждый раз получает ошибку.

В работе \cite{4} выделен ряд шаблонов, связанных с выполнением пользователем поставленных задач, например, шаблон «Отмена действия», когда пользователь отменяет действие сразу после его выполнения, или шаблон «Повторение действий», когда пользователь часто повторяет простые действия (клики мыши или нажатие клавиш). Наличие второго шаблона может означать недостаточную отзывчивость интерфейса, которая ошибочно приводит пользователя к мысли, что система не распознает его действие.

Отдельные исследователи предлагают отслеживать более простые индикаторы: количество вызовов онлайн-справки, количество действий отмены, частое открытие-закрытие выпадающих списков, нажатие одной и той же кнопки более одного раза и т.д. \cite{5}. Другие исследователи основываются на обнаружении проблем поиска информации пользователем в процессе просмотра веб-сайта \cite{6}. Например, выделяется шаблон вертикального или горизонтального перемещения курсора мыши. В процессе визуального поиска на странице пользователь обычно перемещает курсор вслед за элементами, а значит, тратит много времени на поиск элемента.

Перечисленные методы поиска шаблонов поведения пользователей имеют много общего с задачей поиска последовательных шаблонов из области интеллектуального анализа данных [7]. В большинстве случаев все шаблоны являются последовательными, варьируются лишь анализируемые события. Однако данные активности пользователей почти всегда представляют собой не короткие транзакции, а большие наборы действий, которые в большинстве случаев невозможно корректно разделить на поднаборы \cite{2,3}.

Поиск последовательных шаблонов давно и активно применяется в области торговли [8]. Поиск наиболее частых наборов позволяет получать
информацию о том, через какой промежуток времени после покупки товара «А» человек наиболее склонен купить товар «Б» или в какой последовательности приобретаются товары. Получаемые закономерности в действиях покупателей можно использовать для персонализации клиентов, стимулирования продаж определенных товаров, управления запасами [8]. Это позволяет, с одной стороны, увеличить продажи, с другой – предложить клиентам товар, который, скорее всего, будет им интересен, а значит, минимизировать их временные затраты на поиск.

%При проектировании пользовательского интерфейса в соответствии со стандартами ГОСТ-2880690 и ISO 9241-11:1998 аналогичным образом требуется максимизировать результативность (точность и полноту достижения пользователем поставленных целей, успешность выполнения промежуточных задач) и эффективность (отношение израсходованных ресурсов к точности и полноте, с которой пользователи достигают поставленных целей).

Как уже отмечалось, одной из возможных причин появления регулярно повторяющихся шаблонов в данных активности пользователей является
наличие ошибок или затруднений при взаимодействии с интерфейсом. В этом случае может наблюдаться снижение и результативности, и эффективности пользователей. Следовательно, уменьшение числа подобных шаблонов снижает риск возникновения ошибок.

Другой возможной причиной наличия повторяющихся шаблонов в данных активности пользователей является потребность выполнения одних и
тех же повторяющихся цепочек действий для выполнения поставленных задач. Закономерно, что автоматизация промежуточных действий уменьшает затраты ресурсов. Следовательно, чем меньше пользователь совершает однотипных цепочек действий, тем меньше он затрачивает ресурсов, а значит, тем эффективнее взаимодействие.

Конечно, при этом отмечается, что повторяющиеся шаблоны могут быть образованы из-за повторяющихся задач, которые либо невозможно или
нецелесообразно автоматизировать, либо являются нормальным корректным поведением \cite{1}. Поэтому требуется понимание семантики шаблонов и конкретных действий.