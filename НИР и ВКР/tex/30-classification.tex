\chapter{Существующие решения}

\section{Математическая модель активности пользовательской активности ПО}
В статье \cite{7} предлагается математическая модель активности пользователей ПО, основанная на теории последовательных шаблонов для предметной области оценки удобства использования.

Модель состоит из следующих элементов:
\begin{itemize}
	\item множество событий с атрибутами;
	\item множество классов событий;
	\item функция классификации событий;
	\item множество сессий до классификации;
	\item множество сессий после классификации и фильтрации;
	\item множество последовательных шаблонов;
	\item множество значений поддержки последовательных шаблонов;
	\item функция преобразовния класса событий в затрачиваемое время.
\end{itemize}

В качестве функции преобразовния класса событий в затрачиваемое время можно использовать метод оценки эффективности интерфейса – GOMS (Goals, Operators, Methods, Selection Rules – Цели, Операторы, Методы, Правила выбора соответственно), который включает в себя модель Keystroke-level Model (KLM) \cite{8}.

Данная модель может найти применение при оценке удобства использования пользовательских интерфейсов и для решения задач повышения эффективности взаимодействия пользователей с ПО.

Имея значения поддержки и затрачиваемого времени для каждого шаблона, эксперт может сконцентрироваться на наиболее значимых из них
для процесса работы пользователей с ПО в целом. Набор шаблонов при этом будет зависеть от целей проводимого анализа.

Далее эксперт может выдвинуть гипотезы о необходимых изменениях в пользовательском интерфейсе для повышения эффективности взаимодействия пользователей с ПО. При принятии решений эксперту необходимо учитывать множество различных факторов: особенности ПО, психологические факторы использования ПО и особенности пользователей.

Изменение пользовательского интерфейса повлечет изменение множеств событий, сессий и последовательных шаблонов, так как изменится
последовательность действий, необходимых для достижения пользователями поставленных целей.

Таким образом, можно утверждать, что задачей эксперта становится переход от текущей модели активности пользователей к новой, с иным составом сессий и шаблонов, следовательно, и иными значениями поддержки шаблонов и затратами времени пользователей.

После внесения изменений в программный интерфейс возможны повторный сбор и анализ данных активности пользователей, что может под-
твердить либо опровергнуть выдвинутую ранее гипотезу.

Недостатком данной модели является использование заранее предопределенных шаблонов.