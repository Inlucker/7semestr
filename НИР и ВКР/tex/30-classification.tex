\chapter{Существующие решения}

\section{Математическая модель пользовательской активности ПО}
В статье \cite{mat_model} предлагается математическая модель активности пользователей ПО, основанная на теории последовательных шаблонов для предметной области оценки удобства использования. Данный способ совмещает анализ временных характеристик с вычислением уровней поддержки последовательных шаблонов.

Модель состоит из следующих элементов:
\begin{itemize}
	\item множество событий с атрибутами;
	\item множество классов событий;
	\item функция классификации событий;
	\item множество сессий до классификации;
	\item множество сессий после классификации и фильтрации;
	\item множество последовательных шаблонов;
	\item множество значений поддержки последовательных шаблонов;
	\item функция преобразовния класса событий в затрачиваемое время.
\end{itemize}

%В качестве функции преобразовния класса событий в затрачиваемое время можно использовать метод оценки эффективности интерфейса – GOMS (Goals, Operators, Methods, Selection Rules – Цели, Операторы, Методы, Правила выбора соответственно), который включает в себя модель Keystroke-level Model (KLM) \cite{11}.

Данная модель может найти применение при оценке удобства использования пользовательских интерфейсов и для решения задач повышения эффективности взаимодействия пользователей с ПО.

Имея значения поддержки и затрачиваемого времени для каждого шаблона, эксперт может сконцентрироваться на наиболее значимых из них
для процесса работы пользователей с ПО в целом. Набор шаблонов при этом будет зависеть от целей проводимого анализа.

Далее эксперт может выдвинуть гипотезы о необходимых изменениях в пользовательском интерфейсе для повышения эффективности взаимодействия пользователей с ПО. При принятии решений эксперту необходимо учитывать множество различных факторов: особенности ПО, психологические факторы использования ПО и особенности пользователей.

Изменение пользовательского интерфейса повлечет изменение множеств событий, сессий и последовательных шаблонов, так как изменится
последовательность действий, необходимых для достижения пользователями поставленных целей.

Таким образом, можно утверждать, что задачей эксперта становится переход от текущей модели активности пользователей к новой, с иным составом сессий и шаблонов, следовательно, и иными значениями поддержки шаблонов и затратами времени пользователей.

После внесения изменений в программный интерфейс возможны повторный сбор и анализ данных активности пользователей, что может под-
твердить либо опровергнуть выдвинутую ранее гипотезу.

Недостатком данной модели является использование заранее предопределенных шаблонов.

\section{Алгоритм получения ассоциативных правил Apriori}
Базовым алгоритмом, применяемым для получения ассоциативных правил, является алгоритм Apriori \cite{34}, автором которого является Ракеш Агравал (Rakesh Agrawal). Алгоритм Apriori использует стратегию поиска в ширину и осуществляет его снизу-вверх, последовательно перебирая кандидатов (рис. \ref{img:Apriori}).

\imgsc{H}{0.7}{Apriori}{Алгоритм Apriori, перебор кандидатов шаблонов.}

Основной особенностью алгоритма можно считать использование свойства антимонотонности, которое гласит, что поддержка любого набора элементов не может превышать минимальной поддержки любого из его подмножеств. Именно благодаря этому свойству достигается скорость перебора, т.к. исключается анализ редких наборов (рис. \ref{img:AprioriGen}).

\imgsc{H}{0.7}{AprioriGen}{Алгоритм AprioriGen, формирование кандидатов шаблонов.}

Существуют различные модификации алгоритма Apriori и иные алгоритмы \cite{35}, значительно оптимизированные под определенные ситуации. Можно утверждать, что применение существующего проработанного аппарата теории последовательных шаблонов позволит реализовать поиск новых (неизвестных ранее) шаблонов взаимодействия пользователей с информационной системой при меньших временных затратах.

\section{Метод оценки эффективности интерфейса GOMS}
Метод GOMS (сокращение от Goals, Operators, Methods and Selection Rules — Цели, Операторы, Методы и Правила выбора) — это семейство методов, позволяющих провести моделирование выполнения той или иной задачи пользователем и на основе такой модели оценить качество интерфейса.

Идея метода заключается в разбиении взаимодействия пользователя с интерфейсом на атомарные физические и когнитивные действия. Обладая знаниями о метриках каждой из таких составляющих, можно делать заключение об эффективности взаимодействия в целом: оценка
эффективности интерфейса сводится к разбиению типовых задач на элементарные действия и сложению метрик каждого из них.

Метод GOMS включает в себя модель Keystroke-level Model (KLM) [11], которая выделяет следующие элементарные задачи и длительность каждой из них (рассчитанные на основе усредненных данных лабораторных испытаний):

\begin{itemize}
	\item K – нажатие на клавишу в зависимости от уровня владения клавиатурой: профессиональный	наборщик – 0.08 сек., эксперт – 0.12 сек., частая работа с текстом – 0.20 сек., продвинутый пользователь – 0.28 сек., неуверенный пользователь – 0.5 сек., не знакомый с клавиатурой – 1.2 сек.;
	\item P – указание курсором мыши на объект – 1.1 сек.;
	\item B – нажатие или отпускание мыши – 0.1 сек.;
	\item M – умственная подготовка, выбор действия – 1.2 сек.;
	\item H – перемещение руки в исходное положение на клавиатуре – 0.4 сек;
	\item R – ожидание ответа системы, зависящее от	времени выполнения системой запрошенной операции.
\end{itemize}

Оценка времени на решение задачи сводится к сложению продолжительностей каждой из простейших составляющих. Например, задача, состоящая из классов (P, P, B), потребует для завершения 2.3 сек. (1.1 сек. + 1.1 сек. + 0.1 сек.).


\section{Классификация}
Рассмотрев вышеописанные методы анализа пользовательской активности, можно их разделить на следующии категории:
\begin{itemize}
	\item нахождение последовательных шаблонов;
	\item сбор и анализ временных характеристик выполнения пользователем действий и промежутков между ними;
	\item вычисление уровней поддержки шаблонов поведения пользователя.
\end{itemize}