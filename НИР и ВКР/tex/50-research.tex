\chapter{Сравнение}

Поскольку рассмотренные методы используются для решения разных задач, то сравнивать их по критериям нецелесообразно. В таблице \ref{tab} приведено сравнение входных и выходных данных для данных методов.

\begin{table}[H]
	\begin{center}
		\caption{Сравнение входных и выходных данных методов}
		\label{tab}
		""\newline
		\begin{tabular}{ | c | c | c | c | }
			\hline
			Метод & Входные данные  & Выходные данные \\ \hline
			\specialcell{Математическая модель \\ пользовательской \\ активности ПО} & \specialcell{Множество событий, \\ функция классификации \\ событий, множество \\ сессий, множество \\ последовательных \\ шаблонов } & \specialcell{Множество \\ значений \\ поддержки \\ последовательных \\ шаблонов} \\ \hline
			\specialcell{Алгоритм получения \\ ассоциативных правил \\ Apriori} & \specialcell{Транзакции с набором \\ элементов и \\ минимальный уровень \\ поддержки} & \specialcell{Ассоциативные \\ правила} \\ \hline
			\specialcell{Алгоритм получения \\ последовательных \\ шаблонов GSP} & \specialcell{База данных с полями: \\ id последовательности, \\ id и время транзакции, \\ набор элементов \\ и минимальный \\ уровень поддержки} & \specialcell{Последовательные \\ шаблоны} \\ \hline
			\specialcell{Метод оценки \\ эффективности \\ интерфейса GOMS} & \specialcell{Последовательность \\ действий} & \specialcell{Длительность \\ выполнения} \\ \hline
		\end{tabular}
	\end{center}
\end{table}