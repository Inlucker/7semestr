\chapter{Аналитическая часть}
Тестирование удобства использования программного обеспечения обычно состоит из двух этапов.
Первый этап заключается в сборе данных о действиях, совершаемых пользователями посредством взаимодействия с графическим интерфейсом программы (движение курсора мыши, нажатие клавиш мыши, нажатие клавиш клавиатуры и т.д.), и характеристиках действий (координаты курсора, частота нажатия, используемые клавиши и т.д.).
Такие данные обозначаются устоявшимся термином «активность пользователей».
Второй этап – анализ этих данных экспертом с целью выявления проблем связанных с удобством использования, что является трудоемкой задачей.
Поэтому, встает вопрос о хотя бы частичной автоматизации этого этапа, для чего требуется наличие соответствующих моделей и алгоритмов.


\section{Классификация}
Методы анализа пользовательской активности можно разделить на следующии категории:
\begin{itemize}
	\item ручные;
	\item автоматические.
\end{itemize}
То как