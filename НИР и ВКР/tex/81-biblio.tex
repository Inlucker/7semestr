\addcontentsline{toc}{chapter}{Список использованных источников}
\UnnamedStructuredChapter{Список использованных источников}
\bibliographystyle{utf8gost705u}
\begin{thebibliography}{1}
	\bibitem{1}
%	Siochi A.C., Ehrich R.W.
%	Computer Analysis of User Interfaces Based on Repetition in Transcripts of User Sessions
%	// ACM Transactions on Information Systems 1991. Т. 9. № 4. С. 309–335.
	Siochi A.C., Ehrich R.W
	Computer Analysis of User Interfaces Based on Repetition in Transcripts of User Sessions.
	ACM Transactions on Information Systems, 1991, vol. 9, no. 4, pp. 309–335.
	
	\bibitem{2}
	Данилов Н.А., Шульга Т.Э.
	Метод построения тепловой карты на основе точечных данных об активности пользователя приложения
	// Прикладная информатика. 2015. Т. 10. № 2. С. 49–58.
	
	\bibitem{3}
%	Danilov N., Shulga T., Frolova N., Melnikova N., Vagarina N., Pchelintseva E.
%	Software usability evaluation based on the user pinpoint activity heat map
%	// Advances in Intelligent Systems and Computing. 2016. Т. 465. С. 217–225.
	Danilov N., Shulga T., Frolova N., Melnikova N., Vagarina N., Pchelintseva E.
	Software usability evaluation based on the user pinpoint activity heat map.
	Advances in Intelligent Systems and Computing, 2016, vol. 465, pp. 217–225.
	
	\bibitem{4}
%	Balbo S., Goschnick S., Tong D., Paris C.
%	Leading Usability Evaluations to WAUTER
%	// Proc. 11th Australian World Wide Web Conf. (AusWeb), Gold Coast, Australia, Southern Cross Univ., 2005. С. 279–290.
	Balbo S., Goschnick S., Tong D., Paris C.
	Leading Usability Evaluations to WAUTER.
	Proc. 11th Australian World Wide Web Conf. (AusWeb), Gold Coast, Australia, Southern Cross Univ., 2005,	pp. 279–290.
	
	\bibitem{5}
%	Swallow J., Hameluck D., Carey T.
%	User interface instrumentation for usability analysis: A case study
%	// CASCON’97, Toronto, Ontario, 1997.
	Swallow J., Hameluck D., Carey T.
	User interface	instrumentation for usability analysis: a case study.
	CASCON’97, Toronto, Ontario, 1997.
	
	\bibitem{6}
%	Shah I.
%	Event patterns as indicators of usability problems.
%	// Jour. of King Saud Univ. 2008. Т. 20. С. 31–43.
	Shah I.
	Event patterns as indicators of usability problems.
	Jour. of King Saud Univ., Comp. and Inform. Sci., 2008, pp. 31–43.
	
	\bibitem{7}
%	Mabroukeh N.R., Ezeife C.I.
%	A taxonomy of sequential pattern mining algorithms.
%	// ACM Computing Surveys (CSUR). 2010.	Т. 43. №. 1. article no. 3.
	Mabroukeh N.R., Ezeife C.I.
	A taxonomy of sequential pattern mining algorithms.
	ACM Computing Surveys (CSUR), 2010,	vol. 43, no. 1, article no. 3.
	
	\bibitem{8}
	Aloysius G., Binu D.
	An approach to products placement in supermarkets using prefixspan algorithm.
	Jour. of King Saud Univ. Comp. and Inform. Sc., 2013, vol. 25, no. 1, pp. 77–87.
	
	\bibitem{11}
%	Card S., Moran T., Newell A.
%	The keystroke-level model for user performance time with interactive systems.
%	// Communications of the ACM. 1980. Т. 23. №. 7. С. 396–410.
	Card S., Moran T., Newell A.
	The keystroke-level model for user performance time with interactive systems.
	Communications of the ACM, 1980, vol. 23, no. 7, pp. 396–410.
	
	\bibitem{mat_model}
	Сытник А.А., Шульга Т.Э., Данилов Н.А., Гвоздюк И.В.
	Математическая модель активности пользователей программного обеспечения.
	// Программные продукты и системы. 2018. Т. 31. № 1. С. 79-84 
	
	\bibitem{34}
	Agrawal R., Imielinski T., Swami A.N.
	Mining Association Rules between Sets of Items in Large Databases
	// Proceedings of the 1993 ACM SIGMOD international conference on Management of data, SIGMOD '93. – Washington, D.C., USA, 1993. – Vol. 22(2). – Pp. 207-216.
	
	\bibitem{35}
	Agrawal R., Srikant R.
	Fast algorithms for mining association rules
	// Proceedings of the 20th International Conference on Very Large Data Bases, VLDB. – Santiago, Chile, September 1994. – Pp. 487-499.
	
	
\end{thebibliography}
