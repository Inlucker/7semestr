\addcontentsline{toc}{chapter}{Список использованных источников}
\UnnamedStructuredChapter{Список использованных источников}
\bibliographystyle{utf8gost705u}
\begin{thebibliography}{1}
	\bibitem{1}
	Siochi A.C., Ehrich R.W.
	Computer Analysis of User Interfaces Based on Repetition in Transcripts of User Sessions
	// ACM Transactions on Information Systems 1991. Т. 9. № 4. С. 309–335.
	
	\bibitem{2}
	Данилов Н.А., Шульга Т.Э.
	Метод построения тепловой карты на основе точечных данных об активности пользователя приложения
	// Прикладная информатика. 2015. Т. 10. № 2. С. 49–58.
	
	\bibitem{3}
	Danilov N., Shulga T., Frolova N., Melnikova N., Vagarina N., Pchelintseva E.
	Software usability evaluation based on the user pinpoint activity heat map
	// Advances in Intelligent Systems and Computing. 2016. Т. 465. С. 217–225.
	
	\bibitem{4}
	Balbo S., Goschnick S., Tong D., Paris C.
	Leading Usability Evaluations to WAUTER
	// Proc. 11th Australian World Wide Web Conf. (AusWeb), Gold Coast, Australia, Southern Cross Univ., 2005. С. 279–290.
	
	\bibitem{5}
	Swallow J., Hameluck D., Carey T.
	User interface instrumentation for usability analysis: A case study
	// CASCON’97, Toronto, Ontario, 1997.
	
	\bibitem{6}
	Shah I.
	Event patterns as indicators of usability problems.
	// Jour. of King Saud Univ. 2008. Т. 20. С. 31–43.
	
	\bibitem{7}
	Сытник А.А., Шульга Т.Э., Данилов Н.А., Гвоздюк И.В.
	Математическая модель активности пользователей программного обеспечения.
	// Программные продукты и системы. 2018. Т. 31. № 1. С. 79-84 
	
	\bibitem{8}
	Card S., Moran T., Newell A.
	The keystroke-level model for user performance time with interactive systems.
	// Communications of the ACM. 1980. Т. 23. №. 7. С. 396–410.
	
\end{thebibliography}
