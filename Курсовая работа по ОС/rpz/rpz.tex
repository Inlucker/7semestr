%% Преамбула TeX-файла

% 1. Стиль и язык
\documentclass[utf8x, 12pt]{G7-32} % Стиль (по умолчанию будет 14pt)

% Остальные стандартные настройки убраны в preamble.inc.tex.
\sloppy

\usepackage{cmap} % Улучшенный поиск русских слов в полученном pdf-файле
\usepackage[T2A]{fontenc} % Поддержка русских букв
\usepackage[utf8]{inputenc} % Кодировка utf8
\usepackage[english,russian]{babel} % Языки: русский, английский
\usepackage{enumitem}
%\usepackage{pscyr} % Нормальные шрифты
\usepackage[figurename=Рисунок]{caption}

\usepackage{hhline}
\usepackage{amsmath}
\usepackage{amsfonts} 

\usepackage{geometry}
\geometry{left=30mm}
\geometry{right=10mm}
\geometry{top=20mm}
\geometry{bottom=20mm}

% Пакет Tikz
\usepackage{tikz}
\usetikzlibrary{arrows,positioning,shadows}

% Произвольная нумерация списков.
\usepackage{enumerate}

% ячейки в несколько строчек
\usepackage{multirow}

% itemize внутри tabular
\usepackage{paralist,array}

\setlength{\parskip}{1ex plus0.5ex minus0.5ex} % разрыв между абзацами
\setlength{\parskip}{1ex} % разрыв между абзацами
\usepackage{blindtext}

\usepackage{graphicx} 
\usepackage{float}

\newcommand{\imgwc}[4]
{
	\begin{figure}[#1]
		\center{\includegraphics[width=#2]{inc/img/#3}}
		\caption{#4}
		\label{img:#3}
	\end{figure}
}
\newcommand{\imghc}[4]
{
	\begin{figure}[#1]
		\center{\includegraphics[height=#2]{inc/img/#3}}
		\caption{#4}
		\label{img:#3}
	\end{figure}
}
\newcommand{\imgsc}[4]
{
	\begin{figure}[#1]
		\center{\includegraphics[scale=#2]{inc/img/#3}}
		\caption{#4}
		\label{img:#3}
	\end{figure}
}

\usepackage{titlesec}
\titleformat{\section}
	{\normalsize\bfseries}
	{\thesection}
	{1em}{}
\titlespacing*{\chapter}{0pt}{-30pt}{8pt}
\titlespacing*{\section}{\parindent}{*4}{*4}
\titlespacing*{\subsection}{\parindent}{*4}{*4}

\usepackage{setspace}
\onehalfspacing % Полуторный интервал

\frenchspacing
\usepackage{indentfirst} % Красная строка

\usepackage{titlesec}
\titleformat{\chapter}{\LARGE\bfseries}{\thechapter}{20pt}{\LARGE\bfseries}
\titleformat{\section}{\Large\bfseries}{\thesection}{20pt}{\Large\bfseries}

\usepackage{listings}
\usepackage{xcolor}

\lstdefinestyle{python}{
	language=python,
	backgroundcolor=\color{white},
	basicstyle=\footnotesize\ttfamily,
	keywordstyle=\color{blue},
	stringstyle=\color{red},
	commentstyle=\color{gray},
	directivestyle=\color{orange},
	numbers=left,
	numberstyle=\tiny,
	stepnumber=1,
	numbersep=5pt,
	frame=single,
	tabsize=4,
	captionpos=b,
	breaklines=true,
	breakatwhitespace=true,
	escapeinside={\#*}{*)},
	morecomment=[l][\color{magenta}]{\#},
	columns=fullflexible
}

\lstdefinestyle{yaml}{
	language=python,
	backgroundcolor=\color{white},
	basicstyle=\footnotesize\ttfamily,
	keywordstyle=\color{green},
	stringstyle=\color{red},
	commentstyle=\color{gray},
	directivestyle=\color{orange},
	numbers=left,
	numberstyle=\tiny,
	stepnumber=1,
	numbersep=5pt,
	frame=single,
	tabsize=4,
	captionpos=b,
	breaklines=true,
	breakatwhitespace=true,
	escapeinside={\#*}{*)},
	morecomment=[l][\color{magenta}]{\#},
	columns=fullflexible
}

\lstdefinestyle{bash}{
	language=bash,
	backgroundcolor=\color{white},
	basicstyle=\footnotesize\ttfamily,
	keywordstyle=\color{green},
	stringstyle=\color{red},
	commentstyle=\color{gray},
	directivestyle=\color{orange},
	numbers=left,
	numberstyle=\tiny,
	stepnumber=1,
	numbersep=5pt,
	frame=single,
	tabsize=4,
	captionpos=b,
	breaklines=true,
	breakatwhitespace=true,
	escapeinside={\#*}{*)},
	morecomment=[l][\color{magenta}]{\#},
	columns=fullflexible
}

\usepackage{pgfplots}
\usetikzlibrary{datavisualization}
\usetikzlibrary{datavisualization.formats.functions}

\usepackage{graphicx}
\newcommand{\img}[3] {
	\begin{figure}[h!]
		\center{\includegraphics[height=#1]{inc/img/#2}}
		\caption{#3}
		\label{img:#2}
	\end{figure}
}
\newcommand{\boximg}[3] {
	\begin{figure}[h]
		\center{\fbox{\includegraphics[height=#1]{inc/img/#2}}}
		\caption{#3}
		\label{img:#2}
	\end{figure}
}

\usepackage[justification=centering]{caption} % Настройка подписей float объектов

\usepackage[unicode,pdftex]{hyperref} % Ссылки в pdf
\hypersetup{hidelinks}

\usepackage{csvsimple}

\newcommand{\code}[1]{\texttt{#1}}

\usepackage{pdflscape}
\usepackage{rotating}
\usepackage[absolute]{textpos}
\usepackage{fancyhdr}

\usepackage{graphicx}   % Пакет для включения рисунков

% С такими оно полями оно работает по-умолчанию:
\usepackage{geometry}
\geometry{right=10mm}
\geometry{left=30mm}
\geometry{bottom=20mm}
\geometry{top=20mm}
\geometry{ignorefoot}% считать от нижней границы текста

% Пакет Tikz
\usepackage{tikz}
\usetikzlibrary{arrows,positioning,shadows}

% Произвольная нумерация списков.
\usepackage{enumerate}

% ячейки в несколько строчек
\usepackage{multirow}

% itemize внутри tabular
\usepackage{paralist,array}

\setlength{\parskip}{1ex plus0.5ex minus0.5ex} % разрыв между абзацами
\setlength{\parskip}{1ex} % разрыв между абзацами
\usepackage{blindtext}

% Центрирование подписей к плавающим окружениям
\usepackage[justification=centering]{caption}

\usepackage{makecell}

% Настройки листингов.
\ifPDFTeX
% 8 Листинги

\usepackage{listings}
\usepackage{wrapfig}
% Значения по умолчанию
\lstset{
  basicstyle= \footnotesize,
  breakatwhitespace=true,% разрыв строк только на whitespacce
  breaklines=true,       % переносить длинные строки
%   captionpos=b,          % подписи снизу -- вроде не надо
  inputencoding=koi8-r,
  numbers=left,          % нумерация слева
  numberstyle=\footnotesize,
  showspaces=false,      % показывать пробелы подчеркиваниями -- идиотизм 70-х годов
  showstringspaces=false,
  showtabs=false,        % и табы тоже
  stepnumber=1,
  tabsize=4,              % кому нужны табы по 8 символов?
  frame=single,
  escapeinside={(*}{*)}, %выделение
  literate={а}{{\selectfont\char224}}1
  {б}{{\selectfont\char225}}1
  {в}{{\selectfont\char226}}1
  {г}{{\selectfont\char227}}1
  {д}{{\selectfont\char228}}1
  {е}{{\selectfont\char229}}1
  {ё}{{\"e}}1
  {ж}{{\selectfont\char230}}1
  {з}{{\selectfont\char231}}1
  {и}{{\selectfont\char232}}1
  {й}{{\selectfont\char233}}1
  {к}{{\selectfont\char234}}1
  {л}{{\selectfont\char235}}1
  {м}{{\selectfont\char236}}1
  {н}{{\selectfont\char237}}1
  {о}{{\selectfont\char238}}1
  {п}{{\selectfont\char239}}1
  {р}{{\selectfont\char240}}1
  {с}{{\selectfont\char241}}1
  {т}{{\selectfont\char242}}1
  {у}{{\selectfont\char243}}1
  {ф}{{\selectfont\char244}}1
  {х}{{\selectfont\char245}}1
  {ц}{{\selectfont\char246}}1
  {ч}{{\selectfont\char247}}1
  {ш}{{\selectfont\char248}}1
  {щ}{{\selectfont\char249}}1
  {ъ}{{\selectfont\char250}}1
  {ы}{{\selectfont\char251}}1
  {ь}{{\selectfont\char252}}1
  {э}{{\selectfont\char253}}1
  {ю}{{\selectfont\char254}}1
  {я}{{\selectfont\char255}}1
  {А}{{\selectfont\char192}}1
  {Б}{{\selectfont\char193}}1
  {В}{{\selectfont\char194}}1
  {Г}{{\selectfont\char195}}1
  {Д}{{\selectfont\char196}}1
  {Е}{{\selectfont\char197}}1
  {Ё}{{\"E}}1
  {Ж}{{\selectfont\char198}}1
  {З}{{\selectfont\char199}}1
  {И}{{\selectfont\char200}}1
  {Й}{{\selectfont\char201}}1
  {К}{{\selectfont\char202}}1
  {Л}{{\selectfont\char203}}1
  {М}{{\selectfont\char204}}1
  {Н}{{\selectfont\char205}}1
  {О}{{\selectfont\char206}}1
  {П}{{\selectfont\char207}}1
  {Р}{{\selectfont\char208}}1
  {С}{{\selectfont\char209}}1
  {Т}{{\selectfont\char210}}1
  {У}{{\selectfont\char211}}1
  {Ф}{{\selectfont\char212}}1
  {Х}{{\selectfont\char213}}1
  {Ц}{{\selectfont\char214}}1
  {Ч}{{\selectfont\char215}}1
  {Ш}{{\selectfont\char216}}1
  {Щ}{{\selectfont\char217}}1
  {Ъ}{{\selectfont\char218}}1
  {Ы}{{\selectfont\char219}}1
  {Ь}{{\selectfont\char220}}1
  {Э}{{\selectfont\char221}}1
  {Ю}{{\selectfont\char222}}1
  {Я}{{\selectfont\char223}}1
}

% Стиль для псевдокода: строчки обычно короткие, поэтому размер шрифта побольше
\lstdefinestyle{pseudocode}{
  basicstyle=\small,
  keywordstyle=\color{black}\bfseries\underbar,
  language=Pseudocode,
  numberstyle=\footnotesize,
  commentstyle=\footnotesize\it
}

% Стиль для обычного кода: маленький шрифт
\lstdefinestyle{realcode}{
  basicstyle=\scriptsize,
  numberstyle=\footnotesize
}

% Стиль для коротких кусков обычного кода: средний шрифт
\lstdefinestyle{simplecode}{
  basicstyle=\footnotesize,
  numberstyle=\footnotesize
}

% Стиль для BNF
\lstdefinestyle{grammar}{
  basicstyle=\footnotesize,
  numberstyle=\footnotesize,
  stringstyle=\bfseries\ttfamily,
  language=BNF
}

% Определим свой язык для написания псевдокодов на основе Python
\lstdefinelanguage[]{Pseudocode}[]{Python}{
  morekeywords={each,empty,wait,do},% ключевые слова добавлять сюда
  morecomment=[s]{\{}{\}},% комменты {а-ля Pascal} смотрятся нагляднее
  literate=% а сюда добавлять операторы, которые хотите отображать как мат. символы
    {->}{\ensuremath{$\rightarrow$}~}2%
    {<-}{\ensuremath{$\leftarrow$}~}2%
    {:=}{\ensuremath{$\leftarrow$}~}2%
    {<--}{\ensuremath{$\Longleftarrow$}~}2%
}[keywords,comments]

% Свой язык для задания грамматик в BNF
\lstdefinelanguage[]{BNF}[]{}{
  morekeywords={},
  morecomment=[s]{@}{@},
  morestring=[b]",%
  literate=%
    {->}{\ensuremath{$\rightarrow$}~}2%
    {*}{\ensuremath{$^*$}~}2%
    {+}{\ensuremath{$^+$}~}2%
    {|}{\ensuremath{$|$}~}2%
}[keywords,comments,strings]

% Подписи к листингам на русском языке.
\renewcommand\lstlistingname{\cyr\CYRL\cyri\cyrs\cyrt\cyri\cyrn\cyrg}
\renewcommand\lstlistlistingname{\cyr\CYRL\cyri\cyrs\cyrt\cyri\cyrn\cyrg\cyri}

\else
\usepackage{local-minted}
\fi

% Полезные макросы листингов.
% Любимые команды
\newcommand{\Code}[1]{\textbf{#1}}


\begin{document}

\frontmatter % выключает нумерацию ВСЕГО; здесь начинаются ненумерованные главы: реферат, введение, глоссарий, сокращения и прочее.

% Команды \breakingbeforechapters и \nonbreakingbeforechapters
% управляют разрывом страницы перед главами.
% По-умолчанию страница разрывается.

% \nobreakingbeforechapters
% \breakingbeforechapters

%% Также можно использовать \Referat, как в оригинале
%\begin{abstract}
%	Титульный лист. Эта страница нужна мне, чтобы не сбивалась нумерация страниц
%	\cite{Dh}
%	\cite{Bayer}
%	\cite{Habr1}
%	\cite{Noise_func}
%	\cite{Ulich}

%Это пример каркаса расчётно-пояснительной записки, желательный к использованию в РПЗ проекта по курсу РСОИ.

%Данный опус, как и более новые версии этого документа, можно взять по адресу (\url{https://github.com/rominf/latex-g7-32}).

%Текст в документе носит совершенно абстрактный характер.
%\end{abstract}
% НАЧАЛО ТИТУЛЬНОГО ЛИСТА
\begin{center}
	\hfill \break
	\textit{
		\normalsize{Государственное образовательное учреждение высшего профессионального образования}}\\ 
	
	\textit{
		\normalsize  {\bf  «Московский государственный технический университет}\\ 
		\normalsize  {\bf имени Н. Э. Баумана»}\\
		\normalsize  {\bf (МГТУ им. Н.Э. Баумана)}\\
	}
	\noindent\rule{\textwidth}{2pt}
	\hfill \break
	\noindent
	\makebox[0pt][l]{ФАКУЛЬТЕТ}%
	\makebox[\textwidth][c]{«Информатика и системы управления»}%
	\\
	\noindent
	\makebox[0pt][l]{КАФЕДРА}%
	\makebox[\textwidth][r]{«Программное обеспечение ЭВМ и информационные технологии»}%
	\\
	\hfill\break
	\hfill \break
	\hfill \break
	\hfill \break
	\normalsize{\bf Р А С Ч Ё Т Н О - П О Я С Н И Т Е Л Ь Н А Я\space\space З А П И С К А}\\
	\normalsize{\bf к курсовой работе на тему:}\\
	\hfill \break
	\large{Мониторинг вызовов функций ядра Linux}\\
	\hfill \break
	\hfill \break
	\hfill \break
	\hfill \break
	\hfill \break	
	\normalsize {
		\noindent
		\makebox[0pt][l]{Студент}%
		\makebox[\textwidth][c]{}%
		\makebox[0pt][r]{{$\underset{\text{(Подипсь, дата)}}{\underline{\hspace{8cm}}}$ \space Пронин А.С.}}
	}\\
	\hfill \break	
	\normalsize {
		\noindent
		\makebox[0pt][l]{Руководитель курсового проекта}%
		\makebox[\textwidth][c]{ ~~~~~~~~      }%
		\makebox[0pt][r]{{$\underset{\text{(Подпись, дата)}}{\underline{\hspace{6cm}}}$ \space Рязанова Н.Ю.}}
	}
	\hfill \break
	\hfill \break
	\hfill \break
	\hfill \break
\end{center}
\hfill \break
\hfill \break
\begin{center} Москва 2022 \end{center}

\thispagestyle{empty} % 
% КОНЕЦ ТИТУЛЬНОГО ЛИСТА


%%% Local Variables: 
%%% mode: latex
%%% TeX-master: "rpz"
%%% End: 

\tableofcontents


%%% Local Variables:
%%% mode: latex
%%% TeX-master: "rpz"
%%% End:

%
%%% Local Variables:
%%% mode: latex
%%% TeX-master: "rpz"
%%% End:


\StructuredChapter{Введение}

Уровень удобства использования программного интерфейса влияет на качество всего ПО в целом. Признаком недостаточного уровня удобства использования является наличие проблем взаимодействия пользователя с пользовательским интерфейсом. Они могут быть связаны либо со сложностью формулирования плана действий (принятия решений, что делать дальше), либо с непониманием ответа системы (как изменения в интерфейсе связаны с выполненными действиями) \cite{1}.

Проблемы взаимодействия в большинстве случаев можно определить по наличию в данных активности пользователей определенных последовательностей действий (шаблонов). Для их обнаружения применяются различные методы анализа собираемых данных – как требующие ручного анализа (например, тепловые карты \cite{2,3}), так и использующие алгоритмы автоматического анализа \cite{1} на основе шаблонов, выявленных исследователями ранее \cite{4, 5, 6}. Автоматический анализ экономит время и деньги, так как эксперты вместо анализа всех данных фокусируют внимание на отдельных областях пользовательского интерфейса,
где были выявлены соответствующие шаблоны.

% На настоящий момент в открытых научных источниках не удалось найти формализованное представление данных активности пользователей ПО. В статье представлена разработанная авторами математическая модель активности пользователей ПО. Эта модель может найти применение при оценке удобства пользовательских интерфейсов. Целями являются максимальная формализация оценки удобства использования и формирование критериев для повышения эффективности взаимодействия пользователей с пользовательским интерфейсом.

\textbf{Цель работы} – провести обзор существующих методов анализа пользовательской активности, сформулировать критерии для их оценки и провести сравнение рассмотренных методов.

\textbf{Задачи работы:}
\begin{itemize}
	\item рассмотреть существующие решения в области анализа пользовательской активности;
	\item классифицировать методы анализа пользовательской активности;
	\item выбрать критерии для их оценки и сравнить.
\end{itemize}


\mainmatter % это включает нумерацию глав и секций в документе ниже

\chapter{Аналитическая часть}
Тестирование удобства использования программного обеспечения обычно состоит из двух этапов. Первый этап заключается в сборе данных о действиях, совершаемых пользователями посредством взаимодействия с графическим интерфейсом программы (движение курсора мыши, нажатие клавиш мыши, нажатие клавиш клавиатуры и т.д.), и характеристиках действий (координаты курсора, частота нажатия, используемые клавиши и т.д.). Такие данные обозначаются устоявшимся термином «активность пользователей». Второй этап – анализ этих данных экспертом с целью выявления проблем связанных с удобством использования, что является трудоемкой задачей. Поэтому, встает вопрос о хотя бы частичной автоматизации этого этапа, для чего требуется наличие соответствующих моделей и алгоритмов.

\section{Шаблоны поведения пользователя}
По мнению многих исследователей (например, авторов \cite{1,4,5,6}), индикатором проблем удобства использования может являться наличие часто повторяемых одинаковых последовательностей действий. Они могут означать, что пользователь пытается достичь цели и каждый раз терпит неудачу. Например, пользователь пытается взаимодействовать с изображением, которое он принял за кнопку \cite{1}, или пользователь нажимает кнопку и каждый раз получает ошибку.

В работе \cite{4} выделен ряд шаблонов, связанных с выполнением пользователем поставленных задач, например, шаблон «Отмена действия», когда пользователь отменяет действие сразу после его выполнения, или шаблон «Повторение действий», когда пользователь часто повторяет простые действия (клики мыши или нажатие клавиш). Наличие второго шаблона может означать недостаточную отзывчивость интерфейса, которая ошибочно приводит пользователя к мысли, что система не распознает его действие.

Отдельные исследователи предлагают отслеживать более простые индикаторы: количество вызовов онлайн-справки, количество действий отмены, частое открытие-закрытие выпадающих списков, нажатие одной и той же кнопки более одного раза и т.д. \cite{5}. Другие исследователи основываются на обнаружении проблем поиска информации пользователем в процессе просмотра веб-сайта \cite{6}. Например, выделяется шаблон вертикального или горизонтального перемещения курсора мыши. В процессе визуального поиска на странице пользователь обычно перемещает курсор вслед за элементами, а значит, тратит много времени на поиск элемента.

Перечисленные методы поиска шаблонов поведения пользователей имеют много общего с задачей поиска последовательных шаблонов из области интеллектуального анализа данных [7]. В большинстве случаев все шаблоны являются последовательными, варьируются лишь анализируемые события. Однако данные активности пользователей почти всегда представляют собой не короткие транзакции, а большие наборы действий, которые в большинстве случаев невозможно корректно разделить на поднаборы \cite{2,3}.

Поиск последовательных шаблонов давно и активно применяется в области торговли [8]. Поиск наиболее частых наборов позволяет получать
информацию о том, через какой промежуток времени после покупки товара «А» человек наиболее склонен купить товар «Б» или в какой последовательности приобретаются товары. Получаемые закономерности в действиях покупателей можно использовать для персонализации клиентов, стимулирования продаж определенных товаров, управления запасами [8]. Это позволяет, с одной стороны, увеличить продажи, с другой – предложить клиентам товар, который, скорее всего, будет им интересен, а значит, минимизировать их временные затраты на поиск.

%При проектировании пользовательского интерфейса в соответствии со стандартами ГОСТ-2880690 и ISO 9241-11:1998 аналогичным образом требуется максимизировать результативность (точность и полноту достижения пользователем поставленных целей, успешность выполнения промежуточных задач) и эффективность (отношение израсходованных ресурсов к точности и полноте, с которой пользователи достигают поставленных целей).

Как уже отмечалось, одной из возможных причин появления регулярно повторяющихся шаблонов в данных активности пользователей является
наличие ошибок или затруднений при взаимодействии с интерфейсом. В этом случае может наблюдаться снижение и результативности, и эффективности пользователей. Следовательно, уменьшение числа подобных шаблонов снижает риск возникновения ошибок.

Другой возможной причиной наличия повторяющихся шаблонов в данных активности пользователей является потребность выполнения одних и
тех же повторяющихся цепочек действий для выполнения поставленных задач. Закономерно, что автоматизация промежуточных действий уменьшает затраты ресурсов. Следовательно, чем меньше пользователь совершает однотипных цепочек действий, тем меньше он затрачивает ресурсов, а значит, тем эффективнее взаимодействие.

Конечно, при этом отмечается, что повторяющиеся шаблоны могут быть образованы из-за повторяющихся задач, которые либо невозможно или
нецелесообразно автоматизировать, либо являются нормальным корректным поведением \cite{1}. Поэтому требуется понимание семантики шаблонов и конкретных действий.
\chapter{Конструкторская часть}

\chapter{Технологическая часть}
\chapter{Экспериментальная часть}

%\chapter{Организационно-экономический раздел}
%\label{cha:econom}
%%% Local Variables:
%%% mode: latex
%%% TeX-master: "rpz"
%%% End:

%\chapter{Промышленная экология и %езопасность}\label{cha:bzd}


%%% Local Variables:
%%% mode: latex
%%% TeX-master: "rpz"
%%% End:


\backmatter %% Здесь заканчивается нумерованная часть документа и начинаются ссылки и
            %% заключение

\StructuredChapter{Заключение}
По итогу проделанной работы была достигнута цель - проведен обзор существующих методов анализа пользовательской активности, сформулированы критерии для их оценки и проведено сравнение рассмотренных методов.

Также были решены все поставленные задачи, а именно:

\begin{itemize}
	\item рассмотрены существующие решения в области анализа пользовательской активности;
	\item классифицированы методы анализа пользовательской активности;
	\item выбраны критерии для их оценки и проведено сравнение.
\end{itemize}

% % Список литературы при помощи BibTeX
% Юзать так:
%
% pdflatex rpz
% bibtex rpz
% pdflatex rpz
\addcontentsline{toc}{chapter}{Список использованных источников}
\UnnamedStructuredChapter{Список использованных источников}
\bibliographystyle{gost780u}

\begin{thebibliography}{9}
	\bibitem{ftrace} 
	Перехват функций в ядре Linux с помощью ftrace
	\\\texttt{https://m.habr.com/post/413241/}
	
	\bibitem{haifa}
	Haifa Linux Club - Networking Lectures
	\\\texttt{http://haifux.org/network.html}
	
	\bibitem{syscall}
	Loadable Kernel Module Programming and System Call Interception
	\\\texttt{https://www.linuxjournal.com/article/4378}
	
	\bibitem{anatomy} 
	М. Джонс 
	\textit{Анатомия загружаемых модулей ядра Linux}.
	\\\texttt{https://www.ibm.com/developerworks/ru/library/l-lkm/index.html}
	
	\bibitem{sources} 
	Исходные коды ядра Linux
	\\\texttt{http://elixir.free-electrons.com}
\end{thebibliography}


%%% Local Variables: 
%%% mode: latex
%%% TeX-master: "rpz"
%%% End: 


\appendix   % Тут идут приложения

\chapter{}
В Листинге \ref{kern_monitor.c} приведен код разработанного модуля

\begin{lstinputlisting}[
	caption={Код модуля},
	label={kern_monitor.c}
	]{src/kern_monitor.c}
\end{lstinputlisting}
	

%\begin{lstinputlisting}[
%	caption={Код модуля},
%	label={kern_monitor.c},
%	linerange={595-657}
%	]{src/kern_monitor.c}
%\end{lstinputlisting}


%\lstinputlisting{src/kern_monitor.c}

%\chapter{Рисунки, поясняющие работу некоторых функций}
%\label{cha:appendix1}

%\begin{figure}
%\centering
%\caption{Картинка в приложении. Страшная и ужасная.}
%\end{figure}
%\begin{figure}[h!]
%	\centering
%	\includegraphics[width=0.6\textwidth]{img/diagram3.png}
%	\caption{Алгоритм работы init-функции}
%	\label{fig:spire02}
%\end{figure}
%\begin{figure}[h!]
%	\centering
%	\includegraphics[width=0.6\textwidth]{img/diagram3.png}
%	\caption{Алгоритм работы функции-обработчика}
%	\label{fig:spire03}
%\end{figure}
%%% Local Variables: 
%%% mode: latex
%%% TeX-master: "rpz"
%%% End: 


%%% Local Variables: 
%%% mode: latex
%%% TeX-master: "rpz"
%%% End: 


\end{document}

%%% Local Variables:
%%% mode: latex
%%% TeX-master: t
%%% End:
